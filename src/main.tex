%! Author = liu
%! Date = 8/5/20

% Preamble
\documentclass{book}

% Packages
\usepackage{amsmath}
\usepackage{unicode-math}
\usepackage{xeCJK}
\usepackage{bm}
\usepackage{times}
%\usepackage{hyperref}
\usepackage{url}
\usepackage{amssymb}
\usepackage{amsthm}
\usepackage[numbers]{natbib}
%\documentstyle[nips14submit_09,times,art10]{article} % For LaTeX 2.09
\usepackage{array}
\newtheorem{proposition}{Proposition}
\newtheorem{lemma}{Lemma}
\newtheorem{corollary}{Corollary}
\newtheorem{theorem}{Theorem}
\newtheorem*{thmnonumber}{Theorem}
\usepackage{graphicx}
\usepackage{hyperref}


\title{电机学提纲}

\author{
刘沁阳,\\
No department\\
No university\\
On earth \\
}

% Document
\begin{document}
    \maketitle

    \tableofcontents

    \chapter*{Preface}
    This is a self-help book open sourced in Github Repo: \url{https://github.com/fokx/self-help-material/}.\\
    Issues are welcomed.\\\\

    出于方便记忆的目的,文中不乏口语化的表达。\\

    This MATERIAL is provided WITHOUT WARRANTY OF ANY KIND,
    EXPRESS OR IMPLIED,
    INCLUDING BUT NOT LIMITED TO THE WARRANTIES OF MERCHANTABILITY,
    FITNESS FOR A PARTICULAR PURPOSE AND NONINFRINGEMENT.
    IN NO EVENT SHALL THE AUTHORS OR COPYRIGHT HOLDERS BE LIABLE FOR ANY CLAIM,
    DAMAGES OR OTHER LIABILITY, WHETHER IN AN ACTION OF CONTRACT,
    TORT OR OTHERWISE, ARISING FROM,
    OUT OF OR IN CONNECTION WITH THE MATERIAL OR THE USE OR OTHER DEALINGS IN THE MATERIAL.


    \chapter{电磁场基础}
    电机中损耗包括:
    \begin{enumerate}
        \item 铜线中损耗(铜耗)
        \item 机械损耗
        \item 铁耗:
        \begin{itemize}
            \item 磁致损耗(70\%,直流下认为没有)$p_{h}=C_{h} f B_{m}^n V$
            \item 涡流损耗(30\%,采用硅钢片后大大减小) $p_{e}=C_{e} \Delta^2 f^2 B_{m}^2 V$ %TODO check delta on book
        \end{itemize}
    \end{enumerate}

    采用硅钢片减小涡流损耗的原理:
    片与片之间绝缘,使电流流动的路径更窄,从而R更大,I更小,$p_{e}$更小。并且掺入的硅增加导磁,减小导电性能。


    \chapter{直流电机}


    \section{结构}
    \begin{enumerate}
        \item 电枢
        \item 励磁绕组
        \item 电枢绕组
    \end{enumerate}
    补偿绕组对称中心在两个相邻磁极的中心线上


    \section{绕组展开图}
    虚槽数$Z_{u}$ = 元件数S = 换向片数K\\
    极距$\tau$:相邻磁极之间的虚槽数,$\tau = {Z_{u} \over {2 p}}$\\
    线圈宽度$y_{1}={Z_{u} \over {2 p}} \mp \epsilon$,优先减\\\\
    单叠绕组$y=1$,双叠绕组$y=2$\\
    $y_{k}=y=1, y_{2}=y_{1} \mp 1$(优先减,减1:右行绕组,加1:左行绕组)\\

    实线:上层边,虚线:下层边\\
    找一虚一实,找对称中心,画点画线\\
    两实线距离 = 换向片宽度 = 电刷宽度 = 磁极间距\\

    每一个线圈连一个换向片,线圈越多,输出转矩和感应电势越稳定\\


    \section{直流电机中的磁场分析}
    电机中的B指的是径向磁密。\\
    \subparagraph*{串励}
    \subparagraph*{并励}
    \subparagraph*{复励}
    积复励/加复励(串联磁动势与并联磁动势同向,增强)\\
    减复励/差复励\\
    平复励(额定点和空载时一样)\\\\
    短复励(短分接法):先并后串\\
    长复励(长分接法):\\

    线负荷/电负荷:$A={{N i_{a}} \over {\pi D_{a}}}$,
    $F_{a} = A x (-{\tau \over 2} < x < {\tau \over 2})$\\

    交轴电枢反应/交磁作用:磁场零点偏移,极下磁通减小\\
    直轴电枢反应:增磁/去磁

    --发电机/电动机 顺/逆电机转向移动电刷,是去磁还是增磁?


    --电枢反应时,总磁通量如何变化?\\
    --正常工作时,由于饱和,极下总磁通减小。
    \section{功率流程}
    发电机:额定功率是输出电功率\\
    电动机:额定功率是输出机械功率($P_{N}=U_{N} I_{N} \eta_{N}$)


    \section{制动}

    \paragraph{能耗制动}
    断电串电阻,R越小制动越快,R大到开路等于没有能耗制动。
    \subparagraph*{能耗制动过程}
    \subparagraph*{能耗制动运行}位能性恒转矩负载

    \paragraph{反接制动(电压反接)}反压串电阻,适用于需要反向启动的设备

    \paragraph{倒拉反转(电势反接)}直接船电阻,斜率很大,进入第IV象限

    \paragraph{回馈制动}$E_{a}>U$


    \section{电机的换向}
    换向:线圈中电流改变\\
    转向:旋转方向改变

    \paragraph{换向的问题}
    换向时,由于:\\
    电流变化率大,产生电抗电势\\
    几何中性线上由于电枢反应$B\neq0$,产生切割电势\\
    电流换向受到阻碍,发生环火(一圈都是火花)。

    \paragraph{帮助换向的措施}
    \begin{enumerate}
        \item 装换向极:让交轴上B为0,去掉切割电势,再增加B,以抵消电抗电势
        \item 装补偿绕组:抵消电枢反应电势,去掉切割电势
        \item 移动电刷位置:把电刷装在$B=0$的位置,没有切割电势
    \end{enumerate}


    \section{复习}
    直流电动机和发电机一般分别设计使用,一般不混用。\\
    单叠绕组:有几个极(2p)就有几条路(2a),每个极下所有元件串联成一条路。\\
    双叠绕组:并联支路数为极数的2倍,是并联绕组。\\
    单波绕组:并联支路数为2。\\
    磁场分布(磁通密度B沿电枢圆周的分布情况):几何中性线是磁场方向的切换线。
    空载运行时只有主磁场,分布是一个平顶的帽型。
    电枢磁场:电枢磁势最强处在几何中性线上,但会凹陷,形成马鞍形波。


    \chapter{变压器}


    \chapter{交流电机的原理}


    \chapter{异步电机}


    \chapter{同步电机}


%    \bibliography{main}
%    \bibliographystyle{plain}

\end{document}
